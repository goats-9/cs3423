\documentclass[journal,12pt,twocolumn]{IEEEtran}
\usepackage{setspace}
\usepackage{gensymb}
\singlespacing
\usepackage[cmex10]{amsmath}
\usepackage{amsthm}
\usepackage{mathrsfs}
\usepackage{txfonts}
\usepackage{stfloats}
\usepackage{bm}
\usepackage{cite}
\usepackage{cases}
\usepackage{subfig}
\usepackage{longtable}
\usepackage{multirow}
\usepackage{enumitem}
\usepackage{mathtools}
\usepackage{tikz}
\usepackage{circuitikz}
\usepackage{verbatim}
\usepackage[breaklinks=true]{hyperref}
\usepackage{tkz-euclide} % loads  TikZ and tkz-base
\usepackage{listings}
\usepackage{color}    
\usepackage{array}    
\usepackage{longtable}
\usepackage{calc}     
\usepackage{multirow} 
\usepackage{hhline}   
\usepackage{ifthen}   
\usepackage{lscape}     
\usepackage{chngcntr}
\DeclareMathOperator*{\Res}{Res}
\renewcommand\thesection{\arabic{section}}
\renewcommand\thesubsection{\thesection.\arabic{subsection}}
\renewcommand\thesubsubsection{\thesubsection.\arabic{subsubsection}}

\renewcommand\thesectiondis{\arabic{section}}
\renewcommand\thesubsectiondis{\thesectiondis.\arabic{subsection}}
\renewcommand\thesubsubsectiondis{\thesubsectiondis.\arabic{subsubsection}}
\renewcommand\thetable{\arabic{table}}
% correct bad hyphenation here
\hyphenation{op-tical net-works semi-conduc-tor}
\def\inputGnumericTable{}                                 %%

\lstset{
language=bash,
frame=single, 
breaklines=true,
columns=fullflexible,
basicstyle=\ttfamily,
}
%\lstset{
%language=tex,
%frame=single, 
%breaklines=true
%}

\DeclareMathOperator*{\argmax}{arg\,max}
\DeclareMathOperator*{\argmin}{arg\,min}
\begin{document}
\newtheorem{theorem}{Theorem}[section]
\newtheorem{problem}{Problem}
\newtheorem{proposition}{Proposition}[section]
\newtheorem{lemma}{Lemma}[section]
\newtheorem{corollary}[theorem]{Corollary}
\newtheorem{example}{Example}[section]
\newtheorem{definition}[problem]{Definition}
\newcommand{\BEQA}{\begin{eqnarray}}
\newcommand{\EEQA}{\end{eqnarray}}
\newcommand{\define}{\stackrel{\triangle}{=}}
\bibliographystyle{IEEEtran}
\providecommand{\mbf}{\mathbf}
\providecommand{\pr}[1]{\ensuremath{\Pr\left(#1\right)}}
\providecommand{\qfunc}[1]{\ensuremath{Q\left(#1\right)}}
\providecommand{\sbrak}[1]{\ensuremath{{}\left[#1\right]}}
\providecommand{\lsbrak}[1]{\ensuremath{{}\left[#1\right.}}
\providecommand{\rsbrak}[1]{\ensuremath{{}\left.#1\right]}}
\providecommand{\brak}[1]{\ensuremath{\left(#1\right)}}
\providecommand{\lbrak}[1]{\ensuremath{\left(#1\right.}}
\providecommand{\rbrak}[1]{\ensuremath{\left.#1\right)}}
\providecommand{\cbrak}[1]{\ensuremath{\left\{#1\right\}}}
\providecommand{\lcbrak}[1]{\ensuremath{\left\{#1\right.}}
\providecommand{\rcbrak}[1]{\ensuremath{\left.#1\right\}}}
\theoremstyle{remark}
\newtheorem{rem}{Remark}
\newcommand{\sgn}{\mathop{\mathrm{sgn}}}
\providecommand{\abs}[1]{\left\vert#1\right\vert}
\providecommand{\res}[1]{\Res\displaylimits_{#1}} 
\providecommand{\norm}[1]{\left\lVert#1\right\rVert}
\providecommand{\mtx}[1]{\mathbf{#1}}
\providecommand{\mean}[1]{E\left[ #1 \right]}   
\providecommand{\fourier}{\overset{\mathcal{F}}{ \rightleftharpoons}}
\providecommand{\system}[1]{\overset{\mathcal{#1}}{ \longleftrightarrow}}
\newcommand{\solution}{\noindent \textbf{Solution: }}
\newcommand{\cosec}{\,\text{cosec}\,}
\providecommand{\dec}[2]{\ensuremath{\overset{#1}{\underset{#2}{\gtrless}}}}
\newcommand{\myvec}[1]{\ensuremath{\begin{pmatrix}#1\end{pmatrix}}}
\newcommand{\mydet}[1]{\ensuremath{\begin{vmatrix}#1\end{vmatrix}}}
\renewcommand{\vec}[1]{\boldsymbol{\mathbf{#1}}}
\def\putbox#1#2#3{\makebox[0in][l]{\makebox[#1][l]{}\raisebox{\baselineskip}[0in][0in]{\raisebox{#2}[0in][0in]{#3}}}}
     \def\rightbox#1{\makebox[0in][r]{#1}}
     \def\centbox#1{\makebox[0in]{#1}}
     \def\topbox#1{\raisebox{-\baselineskip}[0in][0in]{#1}}
     \def\midbox#1{\raisebox{-0.5\baselineskip}[0in][0in]{#1}}

\vspace{3cm}
\title{Overview of Syntax Analyzer}
\author{Gautam Singh\\CS21BTECH11018}
\maketitle
\tableofcontents
\bigskip

This document provides a brief overview, implementation details and challenges
in creating the given syntax analyzer, which lexes a program in a C-like 
source language, writes the list of tokens encountered along with
their type in one file, while also parsing statement by statement and reporting
the types of statements parsed.

\section{Building and Usage}

To build the syntax analyzer from the source code, enter the following commands 
at a terminal window in the directory where the source files are present.

\begin{lstlisting}
flex lex.l
bison -d -by parser.y
gcc -O2 -lfl -ly lex.yy.c y.tab.c
\end{lstlisting}

This will generate an executable file \texttt{a.out}, which is the syntax
analyzer. To run the syntax analyzer on any select testcase, enter the
following.

\begin{lstlisting}
./a.out src.clike tokens.txt statements.parsed
\end{lstlisting}

where paths of the files are relative to the executable.

\section{Implementation Details}
The following are some important implementation details of the syntax analyzer.
The comments in the source code provide more details.
\begin{enumerate}
    \item The lexical analyzer analyzes the source file token-by-token and
    passes the information on to the syntax analyzer for statement-level
    parsing. This is achieved using the \texttt{\%token} directive in
    \texttt{bison}.
    \item Both the lexical analyzer and syntax analyzer share the following
    variables, with the \texttt{extern} modifier.
    \begin{enumerate}
        \item \texttt{FILE *tfile}: The filestream to write the tokens.
        \item \texttt{FILE *pfile}: The filestream to write the parsed
        statements along with their types.
        \item \texttt{int yyerrfl}: A flag variable to indicate error in
        parsing.
        \item \texttt{int yytype}: A variable indicating the type of statement
        parsed, based on the last character of that statement. This is to handle
        reporting of incorrect syntax at the end of the parsed statement.
    \end{enumerate}
    \item It is much more convenient to handle necessary parts of the file
    output in the lexical analysis stage (such as error diagnostics) as it
    considers the source file token-by-token, rather than in the syntax
    analyzer, where source code is considered phrase-by-phrase.
    \item The syntax analyzer also handles errors that are not specific to one
    statement. As an example, incorrectly matched braces can be reported only by
    the syntax analyzer.
\end{enumerate}

\section{Implementation Issues}
Some of the implementation issues and challenges faced in creating the syntax
analyzer are listed below.
\begin{enumerate}
    \item The syntax analyzer did not handle missing braces and count return
    statements, as shown in public test case 1. This was fixed by adding error
    productions to the grammar for the start variable \texttt{program\_body}.
    The global variable \texttt{yyretcnt} maintained a count of return
    statements with corresponding error productions in the grammar. 
    \item Public test case 2 tests if the grammar parses nested expressions and
    predicates, along with their unary operators properly. Especially for
    predicates, the grammar had to be modified to introduce right associativity
    so that shift-reduce and reduce-reduce conflicts were avoided. Another
    subtle detail exposed by this public test case is the handling of multiple
    statements in the same line, as in C. Redefining \texttt{yytype} enabled
    better error diagnostics for the \emph{correct} statement in the line, and
    not report the error at the end of the line.
    \item Public test case 3 checks the for loop construct, whether empty scopes
    are allowed, and whether statements are allowed outside the scope of a
    block. An extra production for \texttt{block\_scope} enabled the handling of
    empty scopes.
\end{enumerate}
\end{document}
